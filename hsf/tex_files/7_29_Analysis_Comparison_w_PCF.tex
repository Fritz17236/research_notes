\section{Analysis: Comparison with Predictive Coding Framework and Gap-Junction Coupling}


Here we compare the self-coupled model with the predictive coding framework (PCF) as defined in Boerlin \& Deneve, 2013. We demonstrate that an assumption in this model leads to a network estimate whose accuracy indefinitely diverges with time. The correction of this assumption leads to an intermittent model featuring membrane voltage coupling we loosely term gap-junction coupling. We show that the gap-junction model rectifies network estimation error introduced by the PCF. We then show that the corrected gap-junction model and the self-coupled mode produce identical results.

\begin{enumerate}

\item \textbf{\textit{The Predictive Coding Framework (PCF):}} The PCF synthesizes a spiking neural network that implements a given dynamical system. It is briefly derived as follows:\\
\\
Assume we are given in dimensionless form
\begin{itemize}
    \item A Linear Dynamical System  $\dot{x}(\xi) = A x(\xi) + B c(\xi)$,  $x \in \mathbf{R}^d$
    
    \item A Decoder Matrix $D \in \mathbf{R}^{d\hspace{1mm} x \hspace{1mm}N}$ specifying The tuning curve of N neurons in d-dimensional space. \\
\\   
Let $o(t) \in \mathbf{R}^{N}$ describe the spike trains whose $j^{th}$ component is given by
\begin{align*}
	o_j(t) \overset{\Delta}{=} \sum_{k=0}^{\infty} \delta(t - t_j^k),
\end{align*} 
where $t_j^k$ is the time of the $k^{th}$ spike of neuron $j$. 
Define the time-varying firing rate of the neurons by 
\begin{align*}
	\frac{d r}{d t}(t) \overset{\Delta}{=} - \tau_s^{-1} r(t) + \tau_s^{-1} o(t),
\end{align*}
where $\tau_s{-1}$ is the decay rate of $r(t)$ given by the inverse synaptic time constant $\tau_s$. For consistency across models, we transform the preceding two equations to dimensionless time via $\xi = \frac{t}{\tau_s} \implies  \tau_s \, d \xi = dt$. This gives
\begin{align}
	\label{eq:analysis:comparison_sc_vs_pcf_vs_gj:pcf_o_def}
	o_j(\xi) \overset{\Delta}{=} \sum_{k=0}^{\infty} \delta(\xi - \xi_j^k),
\end{align}
where $\xi_j^k$ is the $k^{th}$ spike of neuron $j$ in dimensionless time, and
\begin{align*}
	\frac{d r}{d t}(t) &= - \tau_s^{-1} r(t) + \tau_s^{-1}o(t),
	\\
	\\
	\implies
	\frac{d r}{\tau_s \, d \xi}(\xi) &= - \tau_s^{-1} r(\xi) + \tau_s^{-1} o(\xi),
	\\
	\\
	\implies
	\frac{dr}{d\xi}(\xi) &= - r(\xi) + o(\xi).
\end{align*}    
Letting $\dot{\left[ \hspace{5mm} \right]}$ denote differentiation w.r.t. dimensionless time $\xi$, we arrive at 
\begin{align}
\label{eq:analysis:comparison_sc_vs_pcf_vs_gj:pcf_r_def}
\dot{r}(\xi) \overset{\Delta}{=} - r(\xi) + o(\xi). 
\end{align}

The network estimate is defined as 
\begin{align}
\label{eq:analysis:comparison_sc_vs_pcf_vs_gj:pcf_xhat_def}
\hat{x}(\xi) \overset{\Delta}{=} D r(\xi),
\end{align}
which gives rise to the network estimation error
\begin{align}
\label{eq:analysis:comparison_sc_vs_pcf_vs_gj:pcf_error_def}
e(\xi) \overset{\Delta}{=} x(\xi) - \hat{x}(\xi).
\end{align}

The network chooses spike times $\xi_j^k$ to greedily optimize the objective function
\begin{align*}
\mathcal{L}(\xi) = ||x(\xi + d\xi) - \hat{x}(\xi + d\xi)||^2.
\end{align*}
The PCF uses regularization on the rate $r(\xi)$ for the sake of biological plausibility. At present we ignore this regularization and note that they can only increase the network estimation error $e$, the sole network objective.
Using an identical approach to the derivation of the self-coupled network in section (\ref{section:derivation:basic_model}), we arrive at 
\begin{align*}
d_j^T 
\left(
	x - \hat{x}
\right)
&= 
\frac{d_j^T d_j}{2}
\end{align*}
where $d_j$ is the $j^{th}$ column of $D$. We define membrane voltage to get the spiking condition:
\begin{align}
\label{eq:analysis:comparison_sc_vs_pcf_vs_gj:pcf_voltage_def}
v_j &\overset{\Delta}{=} d_j^T (x - \hat{x}) 
\notag
\\
\\
\notag
\implies
d_j^T e &= v_{th}, 
\end{align}

where $v^{th} = \frac{d_j^T d_j}{2}$.

Deriving the dynamics, the preceding equation defines voltage, which in matrix form is given by
\begin{align*}
V &= D^T 
\left(
	x - \hat{x}
\right)
%
\\
\\
%
\implies
\dot{V}
&= 
D^T \dot{x} - D^T \dot{\hat{x}}
&
\\
\\
%
&= D^T 
\left(
	A x + B c
\right)
 - D^T 
 \left(
 D \dot{r}
 \right)
 %
 \\
 \\
 %
 &= 
 D^T A x
 + D^T B c
 - D^T D
\left(
	-r + o 
\right) 
 .
\end{align*}
The PCF makes the assumption that when the network performs correctly, $x = \hat{x}$. We later quantify the estimation error introduced by this assumption and correct it to form the gap-junction model. For now make the assumed substitution $x = \hat{x} = Dr$. 

\begin{align*}
\dot{V} &= D^T A \left(D r\right) + D^T B c + D^T D r - D^T D o
%
\\
\\
%
&= 
D^T
\left(
	A + I 
\right)
 D r
+
D^T B c 
- D^T D o. 
\end{align*}

The model is finalized by the addition of a voltage leakage term to ensure stability, giving the final dynamics equation

\begin{align}
\label{eq:analysis:comparison_sc_vs_pcf_vs_gj:pcf_voltage_dynamics}
\dot{V} = -v
+ D^T 
\left(
A + I
\right)
D r
+ 
D^T B c
- D^T D o.
\end{align}

Equation (\ref{eq:analysis:comparison_sc_vs_pcf_vs_gj:pcf_voltage_dynamics}) scales the spike train $o_j$ by $d_j^T d_j$. Thus the spiking behavior is described by
\begin{align}
\label{eq:analysis:comparison_sc_vs_pcf_vs_gj:pcf_spiking_behavior}
    &v_{th} = \frac{ d_j^T d_j }{2} \notag \\
    \notag \\
    &\text{if  } v_j > v^{th}_j,\notag \\
    \\
    &\text{then  } v_j^{'} = v_j - d_j^T d_j \int \delta(\tau)  \, d\tau ,\notag \\
    \notag \\ 
    &\text{and  } r_j^{'} = r_j + d_j \int \delta(\tau)  \, d\tau \notag.
\end{align}
Equations (\ref{eq:analysis:comparison_sc_vs_pcf_vs_gj:pcf_voltage_dynamics}) and (\ref{eq:analysis:comparison_sc_vs_pcf_vs_gj:pcf_spiking_behavior}) specify the PCF model we compare against. 

% Simulate PCF model here, display error vs time for long-term evolution, highlight this divergence


\item \textbf{\textit{The Gap-Junction Correction:}} Here we correct the assumption that $\hat{x} = x$ made in the PCF model. We restart the previous derivation from this point and derive more a accurate form of equation (\ref{eq:analysis:comparison_sc_vs_pcf_vs_gj:pcf_voltage_dynamics}) termed the gap-junction model. We also explicitly derive the error made by this assumption and show that this error is accounted for by the gap-junction model. \\

The derivation is identical as the PCF until we derive the voltage dynamics.

\begin{align*}
\dot{V} &= 
D^T A x
+
D^T B c
+ D^T D r
- D^T D o.
\end{align*}
Instead of assuming $x = \hat{x}$, we apply the definition of voltage, equation (\ref{eq:analysis:comparison_sc_vs_pcf_vs_gj:pcf_voltage_def}) in matrix form.
\begin{align*}
v_j &= d_j^T e 
%
\\
\\
%
\implies 
V &= D^T e
%
\\
\\
%
&= 
D^T 
\left(
x - \hat{x}
\right)
%
\\
\\
%
\implies
x 
&=
D^{T \dagger} V  + \hat{x}
%
\\
\\
%
&=
D^{T \dagger} V + D r,
\end{align*}
where $D^{T \dagger}$ is the left Moore-Penrose pseudo-inverse of $D^T$.
Substitute this for $x$ in $\dot{V}$ above to get
\begin{align}
\label{eq:analysis:comparison_sc_vs_pcf_vs_gj:gj_voltage_dynamics}
\dot{V} &= 
D^T A
\left(
	D^{T \dagger} V + D r
\right)
+ D^T D r
+
D^T B c
- D^T D o 
\notag
% 
\\ \notag
\\ 
%
\implies
\dot{V}
&= 
D^T A
D^{T \dagger} V 
+
D^T
\left(
	A + I 
\right)
D r
+ 
D^T B c
- D^T D o.
\end{align}

Equation (\ref{eq:analysis:comparison_sc_vs_pcf_vs_gj:gj_voltage_dynamics}) in conjunction with an identical spiking rule from PCF, equation (\ref{eq:analysis:comparison_sc_vs_pcf_vs_gj:pcf_spiking_behavior}) specifies the gap junction model.
\clearpage

The PCF and gap-junction models differ only in their voltage dynamics equations, (\ref{eq:analysis:comparison_sc_vs_pcf_vs_gj:pcf_voltage_dynamics}) and (\ref{eq:analysis:comparison_sc_vs_pcf_vs_gj:gj_voltage_dynamics}) respectively. Suppose we use both models to simulate a target dynamical system. Denote the estimate of the PCF network as $\hat{x}_{pcf}$, and the estimate of the gap-junction network as $\hat{x}_{gj}$. We wish to derive the difference between these estimates denoted by $\chi$. 
\begin{align}
	\label{eq:analysis:comparison_sc_vs_pcf_vs_gj:pcf_vs_gj_estimation_error_def}
	\chi \hspace{2mm} \overset{\Delta}{=} \hspace{2mm} \hat{x}_{pcf} - \hat{x}_{gj}.
\end{align}

To compute $\chi$, subtract the right hand sides of equation (\ref{eq:analysis:comparison_sc_vs_pcf_vs_gj:gj_voltage_dynamics}) from (\ref{eq:analysis:comparison_sc_vs_pcf_vs_gj:pcf_voltage_dynamics}) to get

\begin{align*}
\dot{V}_{pcf} - \dot{V}_{gj} 
&=
-V_{pcf} - D^T A
D^{T \dagger} V_{gj}
\end{align*}

\end{itemize}










\end{enumerate}














