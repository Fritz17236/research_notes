\section{Extensions of the Basic Model}
The basic model presented in section (\ref{section:derivation:basic_model}) currently has the following limitations
\begin{enumerate}
\item Dynamical systems with complex eigenvalues are ill-defined
\item Only $2d$ of $N$ neurons have nontrivial dynamics. 
\item Network dynamics are inherently periodic and deterministic, not asynchronous. 
\end{enumerate}

We extend the basic model to address these limitations. 

\subsection{Dynamical Systems with Complex Eigenvalues}
Recall the basic self-coupled network equations:

$$
\dot{v}
= 
\begin{bmatrix}
\Lambda & 0
\\
0 & 0
\end{bmatrix}
v +
\begin{bmatrix}
S \left(\Lambda + I_d \right) S & 0
\\
0 & 0
\end{bmatrix}
  \rho - 
 \begin{bmatrix}
S^2 & 0
\\
0 & 0
\end{bmatrix}
    \tilde{o},
$$

$$
\dot{\rho} = -\rho + \tilde{o},
$$

$$
\hat{y} = \begin{bmatrix}
S & 0
\end{bmatrix}
\rho.
$$

When the dynamical system $\dot{x} = Ax + Bc$ is oscillatory, the eigenvalues $\Lambda$ of $A=\mathcal{U} \, \Lambda \, \mathcal{U}^T$ are complex valued. In this case, $\dot{v}$ is a system of complex-valued differential equations. Membrane voltages are physical, necessarily real quantities. To resolve this contradiction, we must address how the self-coupled network manages complex-valued voltages. 

Viewing $A$ as an operator between vector spaces, when $A$ has real eigenvalues, our dynamical system is an operator mapping $\mathbf{R}^{d}$ to itself.  When $A$ contains complex eigenvalues, our dynamical system now maps $\mathbf{C}^{d}$ to itself. To reflect this view, we rewrite the factorization of $A$ so that the eigenvectors form a complex basis set of vectors. Before with real eigenvalues, we wrote $A$ as 
$$
A = \mathcal{U} \Lambda \mathcal{U}^T = 
\begin{bmatrix}
\mathcal{U}_1 & \hdots  & \mathcal{U}_d
\end{bmatrix}
\begin{bmatrix}
\Lambda_1 & 0 & \hdots &0  & 0
\\
0 & \Lambda_2 & 0 & \hdots & 0
\\
\vdots & & & & \vdots
\\
 0& 0  & 0 & 0 & \Lambda_d
 \end{bmatrix}
 \begin{bmatrix}
\mathcal{U}_1^T \\ \vdots  \\ \mathcal{U}_d^T
\end{bmatrix}.
$$

For the case that $\Lambda_j = a_j + ib_j \in \mathbf{C},$ we have

\begin{align*}
\Lambda_j \mathcal{U}_j 
&=
\left(a_j + ib_j\right) \mathcal{U}_j
\\
\\
&= 
\left(a_j \mathcal{U}_j + ib_j \mathcal{U}_j\right)
\\
\\
&=
\frac{||a_j \mathcal{U}_j + ib_j \mathcal{U}_j||}
{||a_j \mathcal{U}_j + ib_j \mathcal{U}_j||}
\left(a_j \mathcal{U}_j + ib_j \mathcal{U}_j\right)
\\
\\
&= 
||a_j \mathcal{U}_j + ib_j \mathcal{U}_j||
\frac
{a_j \mathcal{U}_j + ib_j \mathcal{U}_j}
{||a_j \mathcal{U}_j + ib_j \mathcal{U}_j||}
\\
\\
&=
||\Lambda_j|| \, W_j,
\end{align*}

where 
$$ W_j \overset{\Delta}{=} \frac
{a_j \mathcal{U}_j + ib_j \mathcal{U}_j}
{||a_j \mathcal{U}_j + ib_j \mathcal{U}_j||}
$$
gives the complex basis vector corresponding to $\mathcal{U}_j$.

We now write $A$ as 
$$
A = 
W \bar{\Lambda} W^*
=
\begin{bmatrix}
W_1 & \hdots  & W_d
\end{bmatrix}
\begin{bmatrix}
||\Lambda_1|| & 0 & \hdots &0  & 0
\\
0 & ||\Lambda_2|| & 0 & \hdots & 0
\\
\vdots & & & & \vdots
\\
 0& 0  & 0 & 0 & ||\Lambda_d||
 \end{bmatrix}
 \begin{bmatrix}
W_1^* \\ \vdots  \\ W_d^*
\end{bmatrix}.
$$

Note that for the case of real-valued $\Lambda$, $W_j = \mathcal{U}_j$ and $||\Lambda_j|| = \Lambda_j$. With the rotated (now complex) basis, we derive the self-coupled network as previously. Rotate the left-basis of $D$:
$$
D = W \begin{bmatrix}
S & 0
\end{bmatrix}
V^T
$$


The state in the rotated basis is 
$$ y = W^* x \in \mathbf{C}^d$$.
\\
\\
The network estimate and estimation error are 
$$
\hat{y} = W^* \hat{x} = W^* W \begin{bmatrix} S & 0 \end{bmatrix} V^T r = \begin{bmatrix} S & 0 \end{bmatrix} \rho,
$$

$$
\epsilon = y-\hat{y} = W^* e,
$$

and the neuron spike trains and slow synaptic feedback are respectively given by
$$
\tilde{o} = V^T o,
$$

$$
\rho = V^T r.
$$

Note that the right basis of $D = W \begin{bmatrix}S & 0 \end{bmatrix} V^T$ remains real-valued, provided that the given  $V \in \mathbf{R}^{N \times N}$.

The rotated target dynamical system is

$$
\dot{y} = \bar{\Lambda} y + \beta \tilde{c},
$$

where 
$$
\beta = W^* B W,
$$

$$
\tilde{c} = W^* c.
$$

We repeat the original derivation until we arrive at the error dynamics 

$$
\dot{\epsilon}
= 
\bar{\Lambda} \epsilon
+
\left( 
\bar{\Lambda} + I
\right)
\begin{bmatrix}
S & 0
\end{bmatrix}
\rho
-
\begin{bmatrix}
S & 0
\end{bmatrix}
\tilde{o}.
$$

To get voltage dynamics, we must map $\epsilon \in \mathbf{C}^d$ to $v \in \mathbf{R}^{N}$. We can't simply scale $v_j = S_j \epsilon_j$ as before because this would give a complex vector. Since the original voltage definition follows from the network's spike-optimization, a natural place to seek a new voltage definition is in optimizing the objective

$$
\mathcal{L}(\xi) = ||  y(\xi + d\xi) - \hat{y}(\xi + d\xi)||^2.
$$

When neuron $j$ spikes, the vector $S_j W_j$ is added to the network estimate so that the objective is

\begin{align*}
\mathcal{L}_{sp} &= ||y - \left(\hat{y} + S_j W_j\right) ||^2.
\end{align*}

Now dealing with complex vectors, recall the definition of the complex vector norm,
$$
|| x \in \mathbf{C}^d|| = \sqrt{\sum_{k=0}^d |x_k|^2},
$$
where $|x_k|^2$ denotes the modulus of $x_k$, i.e, $|x_k|^2 = x_k^*x_k$.
Therefore,

\begin{align*}
||y - \left(\hat{y} + S_j W_j\right)||^2
&=
\sum_{k=0}^d \left(y_k - \hat{y}_k - S_j W_{kj} \right)^*\left(y_k - \hat{y}_k - S_j W_{kj} \right),
\end{align*}
where $W_{kj}$ is the $k^{th}$ element of $W_j$, which is the $j^{th}$ column of $W \in \mathbf{C}^{d \times N}$.
The summand is the squared modulus of a complex number, i.e. 
\begin{multline*}
\left(y_k - \hat{y}_k - S_j W_{kj} \right)^*\left(y_k - \hat{y}_k - S_j W_{kj} \right)
= 
\Re{\left\{y_k - \hat{y}_k -S_j W_{kj}\right\}}^2 + \Im{\left\{y_k - \hat{y}_k -S_j W_{kj}\right\}}^2
\\
\\
=
\Re{\left\{ y_k \right\}}^2 + \Re{\left\{ \hat{y}_k \right\}}^2 + \Re{\left\{ S_jW_{kj}\right\}}^2 - 2 \Re{\left\{ y_k \right\}}\Re{\left\{ \hat{y}_k \right\}} -2\Re{\left\{ S_j W_{kj} \right\}}\left[\Re{\left\{ y_k \right\}}-\Re{\left\{ \hat{y}_k \right\}}\right]
\\
\\ 
+
\Im{\left\{ y_k \right\}}^2 + \Im{\left\{ \hat{y}_k \right\}}^2 + \Im{\left\{ S_jW_{kj}\right\}}^2 - 2 \Im{\left\{ y_k \right\}}\Im{\left\{ \hat{y}_k \right\}} -2\Im{\left\{ S_j W_{kj} \right\}}\left[\Im{\left\{ y_k \right\}}-\Im{\left\{ \hat{y}_k \right\}}\right]
\\
\\
=
\Re{\left\{y_k - \hat{y}_k\right\}}^2 + \Im{\left\{y_k - \hat{y}_k\right\}}^2
+ \Re{\left\{ S_j W_{kj} \right\}}^2 + \Im{\left\{ S_j W_{kj} \right\}}^2 -2\Re{\left\{ S_j W_{kj} \right\}}\left[\Re{\left\{ y_k \right\}}-\Re{\left\{ \hat{y}_k \right\}}\right]-2\Im{\left\{ S_j W_{kj} \right\}}\left[\Im{\left\{ y_k \right\}}-\Im{\left\{ \hat{y}_k \right\}}\right].
\end{multline*}

The first two terms give the modulus $|y_k - \hat{y}_k|$. For the remaining terms group real parts together and complete the square, then do likewise with the imaginary terms to get:
\begin{multline*}
\Re{\left\{ S_j W_{kj} \right\}}^2  -\Re{\left\{ S_j W_{kj} \right\}}\left[\Re{\left\{ y_k \right\}}-\Re{\left\{ \hat{y}_k \right\}}\right]+ \Im{\left\{ S_j W_{kj} \right\}}^2-\Im{\left\{ S_j W_{kj} \right\}}\left[\Im{\left\{ y_k \right\}}-\Im{\left\{ \hat{y}_k \right\}}\right] =
\\
\left(\Re\left\{S_j W_{kj} \right\} - \Re\left\{ y_k-\hat{y}_k\right\} \right)^2 -  \Re\left\{ y_k-\hat{y}_k
\right\}^2
+
\left(\Im\left\{S_j W_{kj} \right\} - \Im\left\{ y_k-\hat{y}_k\right\} \right)^2 -  \Im\left\{ y_k-\hat{y}_k
\right\}^2
\\
=
\left|S_j W_{kj}-y_k-\hat{y}_k\right|^2 - \left|y_k-\hat{y}_k\right|^2.
\end{multline*}

All together the squared modulus of the summand is

\begin{align*}
\left(y_k - \hat{y}_k - S_j W_{kj} \right)^*\left(y_k - \hat{y}_k - S_j W_{kj} \right)
&=
\left|y_k - \hat{y}_k\right|^2 + \left|S_j W_{kj}- y_k-\hat{y}_k\right|^2 - \left|y_k-\hat{y}_k\right|^2. 
\end{align*}

It follows that 

\begin{align*}
||y - \left(\hat{y} + S_j W_j\right)||^2 
&=
\sum_{k=0}^{d} \left|S_j W_{kj}- y_k-\hat{y}_k\right|^2 
\\
\\
&=
||S_j W_j-\epsilon||^2 .
\end{align*}

The spike condition then is

\begin{align*}
\mathcal{L}_{sp} < \mathcal{L}_{ns}
\\
\\
\implies
||S_j W_j-\epsilon||^2 < ||\epsilon||^2.
\end{align*}
Simplify:
\begin{align*}
||S_j W_j-\epsilon||^2
&=
\sum_{k=0}^{d} 
\Re\left\{S_j^2 W_{kj}^2 - 2 S_j W_{kj}\epsilon_k + \epsilon_k^2\right\}
+
\Im\left\{S_j^2 W_{kj}^2 - 2 S_j W_{kj}\epsilon_k + \epsilon_k^2\right\}
\\
\\
&=
\sum_{k=0}^{d} 
\Re\left\{S_j^2 W_{kj}^2 - 2 S_j W_{kj}\epsilon_k\right\}
+
\Im\left\{S_j^2 W_{kj}^2 - 2 S_j W_{kj}\epsilon_k\right\} + ||\epsilon||^2.
\\
\\
&=
||S_jW_j||^2
-2 \, \sum_{k=0}^{d} 
\left(
\Re\left\{S_j W_{kj}\epsilon_k\right\}
+
\Im\left\{S_j W_{kj}\epsilon_k\right\}
\right)
 + ||\epsilon||^2.
 \\
 \\
 &= 
||S_jW_j||^2
-2 
\left(
\Re \left\{S_jW_j^*\epsilon\right\}
+
\Im \left\{S_jW_j^*\epsilon\right\}
\right)
 + ||\epsilon||^2.
\end{align*}

The spiking condition is therefore 
\begin{align*}
\Re \left\{S_jW_j^*\epsilon\right\}
+
\Im \left\{S_jW_j^*\epsilon\right\}
>
\frac{||S_jW_j||^2}{2}.
\end{align*}

Once again as a sanity check, for the case of only real numbers, we recover the original voltage threshold equations. 

If we split the real and imaginary components of $W_j$ into sepate vectors, we can write this expression as

$$
S_j \left( \Re \left\{W_j^* \epsilon\right\} + \Im \left\{W_j^*\epsilon\right\} \right) > \frac{||S_j W_j||^2}{2}.
$$

This suggests the voltage definition
$$
v = 
\begin{bmatrix}S \\ 0\end{bmatrix} \left( \Re\left\{W^*\epsilon\right\} + \Im\left\{W^*\epsilon\right\}\right) \in \mathbf{R}^d
$$
Put simply, the voltage is the sum of imaginary and real parts.


With this definition, we can resume the derivation as before from the error dynamics:

\begin{align*}
\dot{\epsilon}
&= 
\bar{\Lambda} \epsilon
+
\left( 
\bar{\Lambda} + I
\right)
\begin{bmatrix}
S & 0
\end{bmatrix}
\rho
-
\begin{bmatrix}
S & 0
\end{bmatrix}
\tilde{o}
\\
\\
\implies 
\dot{v} &= \begin{bmatrix}S \\ 0\end{bmatrix} \left( \Re\left\{W^*\dot{\epsilon}\right\} + \Im\left\{W^*\dot{\epsilon}\right\}\right)
\\
\\
&=
\begin{bmatrix}S \\ 0\end{bmatrix} \left( \Re\left\{W^*
\bar{\Lambda} \epsilon
+
\left( 
\bar{\Lambda} + I
\right)
\begin{bmatrix}
S & 0
\end{bmatrix}
\rho
-
\begin{bmatrix}
S & 0
\end{bmatrix}
\tilde{o}
\right\} + \Im\left\{W^*
\bar{\Lambda} \epsilon
+
\left( 
\bar{\Lambda} + I
\right)
\begin{bmatrix}
S & 0
\end{bmatrix}
\rho
-
\begin{bmatrix}
S & 0
\end{bmatrix}
\tilde{o}
\right\}\right)
\\
\\
&=
\begin{bmatrix}
S \\ 0
\end{bmatrix} \left(
\Re\left\{W^*
\bar{\Lambda} \epsilon\right\} + \Im \left\{W^*
\bar{\Lambda} \epsilon\right\}
\right)
+
\begin{bmatrix}
S
\left( 
\bar{\Lambda} + I
\right)
S & 0
\\
0 & 0
\end{bmatrix}
\rho
-
\begin{bmatrix}
S^2 & 0
\\ 0 & 0
\end{bmatrix}
\tilde{o}.
\end{align*}

To summarize, the self-coupled network model is extended to complex-valued dynamical systems by the following:
The rotated basis is no


