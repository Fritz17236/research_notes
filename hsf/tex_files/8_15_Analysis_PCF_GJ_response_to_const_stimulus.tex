\section{Analysis: PCF  and Gap-Junction Response to Constant Stimulus}

 We compare the network estimate of all three models (PCF, gap-junction, and self-coupled) for the case of a constant driving stimulus. 
 
 Let all 3 models have the same parameters as given by equation (\ref{eq:analysis:comparison_sc_vs_pcf_vs_gj:pcf_demo_sim_params}) with the exception that 

\begin{align*}
	c(\xi) &= c = 
	\begin{bmatrix}
		1 \\ 0	
	\end{bmatrix}, 
\end{align*}
and
$x(0) = \begin{bmatrix} \frac{1}{2} & 0 \end{bmatrix}$. 
 
\begin{enumerate}
 
\item \textbf{\textit{PCF Network Response to Constant Stimulus:}}

From equation (\ref{eq:analysis:comparison_sc_vs_pcf_vs_gj:pcf_voltage_dynamics}), the PCF dynamics become
\begin{align*}
	\dot{V}_{pcf}
	&=
	- V_{pcf}
	+
	D^T 
	\left(
		-I + I
	\right)
	D^T r
	+
	D^T 
	\begin{bmatrix}
		1 \\ 0
	\end{bmatrix}
	-
	D^T D o
	%
	\\
	\\
	%
	&= 
	-V_{pcf}
	+ 
	D^T 
	\begin{bmatrix}
		1 \\ 0
	\end{bmatrix}
	-
	D^T D o.
\end{align*}
All voltages are initially 0. From equation (\ref{eq:analysis:comparison_sc_vs_pcf_vs_gj:pcf_spiking_behavior}) the thresholds are identically $\frac{1}{2}$. Until the first spike, neuron $j$'s voltage integrates the quantity $d_j^T 	\begin{bmatrix}	1 \\ 0	\end{bmatrix}$. Denote the neuron $j$ whose tuning curve $d_j$ is closest in angle to $c$ by 

\begin{align*}
	j_{max} \overset{\Delta}{=}
	 \underset{
	 	i \in 
	 	\left[
	 		1, \ldots, N
	 	\right]
	 }
	 {argmax}
	 \hspace{4mm}
	 	d_j^T c.
\end{align*}
Neuron $j_{max}$ will receive the highest driving force and will therefore reach its threshold  before any other neuron. It will then be reset by $1$ to $- \frac{1}{2}$. Each other neuron $k$ will also be reset (decremented) by $d_k^T d_{j_{max}}$, proportional to their angle relative to both neuron $j_{max}$ and the driving strength $c$. This sequence will repeat periodically so that only neuron $j_{max}$ fires at a constant rate. 

We write the PCF network as the one-dimensional equation

\begin{align*}
	v_{pcf} &= 
	- v_{pcf}
	+ d_{j_{max}}^T c 
	- o_{j_{max}}.
\end{align*}

This is a form of the leaky integrate-and-fire (LIF) model, with drive term $d_j^T c(\xi)$. The neuron is driven by inner product $d_{j_{max}}^T c$. Note from equation (\ref{eq:analysis:comparison_sc_vs_pcf_vs_gj:pcf_spiking_behavior}) that the threshold voltage varies with $||d_{j_{max}}||^2$.  For clarity, we drop the subscripts $j$, $j_{max}$ in the following equations. It is understood that we are referring to the solely spiking neuron $j_{max}$.
With initial condition $v_{pcf}(0) = - \frac{||d||^2}{2}$, the neuron's trajectory is integrated as 
\begin{align}
\label{eq:analysis:comparison_sc_vs_pcf_vs_gj:const_stim:pcf_voltage_trajectory}
	v_{pcf}(\xi)
	&= 
	d^T c - e^{-\xi} 
	\left(
		d^T c + \frac{||d||^2}{2}
	\right). 
\end{align}

The neuron spikes when it reaches the threshold $v_{pcf} = ||d||^2$. To compare with the self-coupled network, we note that the singular value associated with neuron $j$ of the decoder matrix $S = ||d||^2$.

From the preceding equation with voltage at  threshold $\frac{||d||^2}{2}$,
\begin{align*}
	\frac{||d||^2}{2}
	&= 
	d^T c - e^{-\xi_{spike}} 
	\left(
		d^T c + \frac{||d||^2}{2}
	\right)
	%
	\\
	\\
	%
	\implies
	e^{-\xi_{spike}}
	&= 
		\frac
	{
		d^T c - \frac{||d||^2}{2}
	}
	{
		d^T c + \frac{||d||^2}{2}
	}
	%
	\\
	\\
	%
	\implies
	\xi_{spike}
	&= 
	ln
	\left(
		d^T c + \frac{||d||^2}{2}
	\right)
	-
	ln
	\left(
		d^T c - \frac{||d||^2}{2}
	\right)		
\end{align*}
This leads to a firing rate 

\begin{align}
\label{eq:analysis:comparison_sc_vs_pcf_vs_gj:const_dynamics:pcf_spike_rate}
	\phi_{pcf}
	\left(
		d
	\right)
	 =
	 \frac
	 {
	 	1
	 }
	 {
		ln
		\left(
			d^T c + \frac{||d||^2}{2}
		\right)
		-
		ln
		\left(
			d^T c - \frac{||d||^2}{2}
		\right)		
	}
\end{align}


A self-coupled neuron spike adds $U_1$ to its network estimate. Using an identical analysis to this case as done in section (\ref{section:analysis:rmse_vs_spike_rate_constant_dynamics}), we substitute $d$ for $U_1$ to arrive at the steady state network estimate of the PCF network:

\begin{align}	\label{eq:analysis:comparison_sc_vs_pcf_vs_gj:const_dynamics:pcf_network_estimate_steady_state}
\hat{x}_{pcf}(\xi)
&= 
\left(
1 + 
\frac
{
	1
}
{
	2 \, d^T c
}
\right)
e^
{
	- \hspace{2mm}
	\left(
		\xi - \xi_1^1
	\right)
	\mod
	{
		\frac
		{
			1
		}
		{
			\phi
		}
	}
}
\, \, d.
\end{align}

\item \textbf{\textit{Gap-Junction Network Response to Constant Stimulus:}} Here we derive the decoded estimate of a gap-junction network driven by a constant stimulus, $c(\xi) = \begin{bmatrix}
1 & 0
\end{bmatrix}
$. All other parameters are identical to those in equation (\ref{eq:analysis:comparison_sc_vs_pcf_vs_gj:pcf_demo_sim_params}).

From the dynamics equation (\ref{eq:analysis:comparison_sc_vs_pcf_vs_gj:gj_voltage_dynamics}) gap-junction voltages are continuously coupled to one another via $D^T A D^{T \dagger}$. We thus need to solve the entire system between spikes rather than reducing it to a single dimension. Let $\tilde{[]}$ denote the Laplace transform of a variable. Assume neuron j has just spiked so that 

\begin{align*}
V(0) = -\frac{1}{2}
\begin{bmatrix}
d_1 ^T d_j
\\
\vdots
\\
d_j^T d_j
\\
\vdots
\\
d_N^T d_j
\end{bmatrix}.
\end{align*}

Since $c(\xi) = \begin{bmatrix}
1 \\ 0
\end{bmatrix}$, $B = I$, and $o(\xi) = 0$ between spikes, we have

$$
	\dot{V} 
	=
	D^T A D^{T \dagger} V
	+ 
	D^T 
	\left(
		A + I
	\right)
	D r
	+
	d_1,
$$
where $d_1$ is the first column of $D$. 
Apply the one-sided Laplace Transform to both sides and use the Laplace derivative property:
\begin{align*}
	s \tilde{V} - V(0)
	&=
	D^T A D^{T \dagger} \tilde{V}
	+
	D^T 
	\left(
		A + I
	\right)o
	D \tilde{r}
	+
	\mathcal{L}\left[d_1\right]
	%
	\\
	\\
	%
	\implies
	\left(
	s \, I - D^T A D^{T \dagger} 
	\right)
	\tilde{V}
	&=
	V(0)
	+
		D^T 
	\left(
		A + I
	\right)
	D \tilde{r}
	+
	\tilde{d_1}	
	%
	\\
	\\
	%
	\implies 
	\tilde{V}
	&= 
	\left(
	s \, I - D^T A D^{T \dagger} 
	\right)^{-1}
	\left[
		V(0)
		+
		D^T 
		\left(
			A + I
		\right)
		D \tilde{r}
		+
		\tilde{d_1}	
	\right]
	%
	\\
	\\
	%
	&= 
	\left(
		sI - D^T A D^{T \dagger}
	\right)^{-1}
	V(0)
	+ 
		\left(
		sI - D^T A D^{T \dagger}
	\right)^{-1}
	D^T 
	\left(
		A + I
	\right)	
	D \tilde{r}
	+
	\tilde(d_1).
\end{align*}

Now apply the inverse Laplace transform. Note that by definition of matrix exponential, 

$$
	\mathcal{L}^{-1} 
	\left(
		sI - D^T A D^{T \dagger}
	\right)^{-1}
	= e^{\xi D^T A D^{T \dagger}}.
$$
Therefore, 
\begin{multline}
\label{eq:analysis:comparison_sc_vs_pcf_vs_gj:gj_voltage_intermittent_laplace_transformed}
	V(\xi)
	=
	e^{\xi D^T A D^{T \dagger}} V(0)
	+
	\mathcal{L}^{-1}
	\left[
		\left(
			sI - D^T A D^{T \dagger}
		\right)^{-1}
		D^T B \tilde{c}
	\right]
	\\
		+ 
	\mathcal{L}^{-1}
	\left[
		\left(
			sI - D^T A D^{T \dagger}
		\right)^{-1}
		D^T 
		\left(
			A + I
		\right)	
		D \tilde{r}
	\right].
\end{multline}
\\
To simplify the second term of equation (\ref{eq:analysis:comparison_sc_vs_pcf_vs_gj:gj_voltage_intermittent_laplace_transformed}), use the convolution-product property of the Laplace transform to get
\\
$$
	\mathcal{L}^{-1}
	\left[
		\left(
			sI - D^T A D^{T \dagger}
		\right)^{-1}
		D^T B \tilde{c}
	\right]
	=
	\mathcal{L}^{-1}
	\left[
			\left(
			sI - D^T A D^{T \dagger}
		\right)^{-1}		
	\right]
	\,	* \, 
		\mathcal{L}^{-1}
	\left[
			D^T B \tilde{c}
	\right].
$$
\\
Note $c(\xi) = \begin{bmatrix}
1 \\ 0
\end{bmatrix}$, and $B = I$. Therefore,
\\
$$
		\mathcal{L}^{-1}
	\left[
			D^T B \tilde{c}
	\right] = D^T B c = d_1,
$$\\
where $d_1$ is the first column of $D$.
The entire second term in  $V(\xi)$ above becomes

$$
	\mathcal{L}^{-1}
	\left[
		\left(
			sI - D^T A D^{T \dagger}
		\right)^{-1}
		D^T B \tilde{c}
	\right]
	=
	e^{\xi D^T A D^{T \dagger}} * d_1.
$$
\\
Evaluating the convolution, bring $d_1$ outside the integral, a linear operator:
$$
	e^{\xi D^T A D^{T \dagger}} * d_1 (\xi)
	=
	\int_{\tau=-\infty}^{\infty}
		e^{
		\left(
			\xi - \tau
		\right)
		 D^T A D^{T \dagger}} 
	\, d\tau
	 \, d_1.
$$
\\
The state $V(\xi)$ depends only on the past up to $V(0)$ so that $0 < \xi - \tau \leq  \xi$: 
\\
$$
	e^{\xi D^T A D^{T \dagger}} * d_1 (\xi)
	=
	\int_{\tau=0}^{\xi}
		e^{
		\left(
			\xi - \tau
		\right)
		 D^T A D^{T \dagger}}  
	\, d\tau
	\, d_1.
$$
\\
The integral of the matrix exponential $\int_{t=0}^{T} \, e^{tX} \, dt= X^{-1} \left(e^{Tx} - I \right)$. Thus,
\\
\begin{equation}
\label{eq:analysis:comparison_sc_vs_pcf_vs_gj:gj_voltage_intermittent_laplace_transformed_term_2}
	\mathcal{L}^{-1}
	\left[
		\left(
			sI - D^T A D^{T \dagger}
		\right)^{-1}
		D^T B \tilde{c}
	\right]
=
\left(
	D^T A D^{T \dagger}
\right)^{-1}
\left(
	e^{\xi D^T A D^{T \dagger}} - I
\right)
d.
\end{equation}

Note the notation $d = D^T B c$. 
Looking at the final term of equation (\ref{eq:analysis:comparison_sc_vs_pcf_vs_gj:gj_voltage_intermittent_laplace_transformed}), assume the network estimate is periodic with period $\frac{1}{\phi}$, where $\phi$ is the unknown spike rate. Between spikes, the dynamics of $r(\xi)$ are known from equation (\ref{eq:analysis:comparison_sc_vs_pcf_vs_gj:pcf_r_def}) solved as
$$
	r(\xi) = e^{-\xi I} r(0), \hspace{4mm} 0 < \xi \leq  \frac{1}{\phi}.
$$
Hence, 
\begin{align*}
	\mathcal{L}^{-1}
	\left[
		\left(
			sI - D^T A D^{T \dagger}
		\right)^{-1}
		D^T 
		\left(
			A + I
		\right)	
		D \tilde{r}
	\right]
	&= 
	\mathcal{L}^{-1}
	\left[
		\left(
			sI - D^T A D^{T \dagger}
		\right)^{-1}		
	\right]
	\, 
	*	
	\,
	\mathcal{L}^{-1}
	\left[
		D^T 
		\left(
			A + I
		\right)	
		D \tilde{r}
	\right]
	%
	\\
	\\
	%
	&=
	e^{\xi D^T A D^{T \dagger}}
	\, 
	*	
	\,
		D^T 
		\left(
			A + I
		\right)	
		D \, 
		e^{-\xi \, I} \, r(0)		
	%
	\\
	\\
	%
	&= 	e^{\xi D^T A D^{T \dagger}}
	\, 
	*	
	\,
	\left(
		D^T A D e^{-\xi \, I} \, r(0)
		+
		D^T D e^{-\xi \, I} \, r(0)
	\right)	
	%
	\\
	\\
	%
	&=
	e^{\xi D^T A D^{T \dagger}}
	\, 	*  \,
	D^T A D e^{-\xi \, I} \, r(0)	
	+
	e^{\xi D^T A D^{T \dagger}}
	\, 	*  \,
	D^T D e^{-\xi \, I} \, r(0).	
\end{align*}

The two convolutions are nearly identical so we solve the simpler of the two:
\\
$$
	e^{\xi D^T A D^{T \dagger}}
	\, 	*  \,
	D^T D e^{-\xi \, I} \, r(0)
	= 
	\int_{\tau=0}^{\xi} \,
		e^{ \left(\xi-\tau\right) D^T A D^{T \dagger}}
		D^T D e^{-\tau \, I} \, r(0)
		\, d\tau.
$$

Note that $e^{-\tau \, I}$ simplifies as 
\begin{align*}
e^{-\tau \, I}
&=
\sum_{k=0}^{\infty}
\frac{\left(-\tau I\right)^{k}}{k!}
%
\\
\\
%
&=
\left(
	\sum_{k=0}^{\infty}
	\frac{(-\tau^{k})}{k!}
\right) I
%
\\
\\
%
&= e^{-\tau} I.
\end{align*}
The scalar and identity matrix can both move to the beginning of the integral and reformed into a matrix:
\begin{align*}
	\int_{\tau=0}^{\xi} \,
		e^{ \left(\xi-\tau\right) D^T A D^{T \dagger}}
		D^T D e^{-\tau \, I} \, r(0)
	\, d\tau
	&= 
	\int_{\tau=0}^{\xi} \,
		e^{-\tau \, I}
		e^{ \left(\xi-\tau\right) D^T A D^{T \dagger}}
		D^T D \, r(0)
	\, d\tau
	%
	\\
	\\
	%
	&=
	e^{\xi D^T A D^{T \dagger}}
	\int_{\tau=0}^{\xi} \,
		e^
		{
			-\tau \, \left( I + D^T A D^{T \dagger} \right)
		}
	\, d\tau
	\, \,
	D^T D \, r(0)
	%
	\\
	\\
	%
	&=
	e^{\xi D^T A D^{T \dagger}}
	\left( I + D^T A D^{T \dagger} \right)^{-1}
	\left(
		e^{\xi \left( I + D^T A D^{T \dagger} \right)} - I
	\right)
	D^T D \, r(0).
\end{align*}

From this expression it follows that 
\\
$$
	e^{\xi D^T A D^{T \dagger}}
	\, 
	*	
	\,
		D^T 
		\left(
			A + I
		\right)	
		D \, 
		e^{-\xi \, I} \, r(0)
	=
		e^{\xi D^T A D^{T \dagger}}
	\left( I + D^T A D^{T \dagger} \right)^{-1}
	\left(
		e^{\xi \left( I + D^T A D^{T \dagger} \right)} - I
	\right)
	D^T(A + I)D \, r(0).	
$$
\\
Hence, 

\begin{align}
\label{eq:analysis:comparison_sc_vs_pcf_vs_gj:gj_voltage_intermittent_laplace_transformed_term_1}
\mathcal{L}^{-1}
	\left[
		\left(
			sI - D^T A D^{T \dagger}
		\right)^{-1}
		D^T 
		\left(
			A + I
		\right)	
		D \tilde{r}
	\right]
	=
	e^{\xi D^T A D^{T \dagger}}
	\left( I + D^T A D^{T \dagger} \right)^{-1}
	\left(
		e^{\xi \left( I + D^T A D^{T \dagger} \right)} - I
	\right)
	D^T(A + I)D \, r(0).	
\end{align}

Using equations (\ref{eq:analysis:comparison_sc_vs_pcf_vs_gj:gj_voltage_intermittent_laplace_transformed_term_2}) and (\ref{eq:analysis:comparison_sc_vs_pcf_vs_gj:gj_voltage_intermittent_laplace_transformed_term_1}), the voltage trajectory equation (\ref{eq:analysis:comparison_sc_vs_pcf_vs_gj:gj_voltage_intermittent_laplace_transformed}) becomes 
\begin{align}
	\label{eq:analysis:comparison_sc_vs_pcf_vs_gj:gj_voltage_solved_full}
	V(\xi)
	= \notag \\ \notag
	& e^{\xi D^T A D^{T \dagger}} V(0)
	\\ \notag
	\\ &+ 
	\left(
		D^T A D^{T \dagger}
	\right)^{-1}
	\left(
		e^{\xi D^T A D^{T \dagger}} - I
	\right)
	d
	\\ \notag
	\\&+ \notag
	e^{\xi D^T A D^{T \dagger}}
	\left( I + D^T A D^{T \dagger} \right)^{-1}
	\left(
		e^{\xi \left( I + D^T A D^{T \dagger} \right)} - I
	\right)
	D^T(A + I)D \, r(0).
\end{align}
\\
In the case $A = -I$, equation (\ref{eq:analysis:comparison_sc_vs_pcf_vs_gj:gj_voltage_solved_full}) simplifies considerably:

\begin{align}	
\label{eq:analysis:comparison_sc_vs_pcf_vs_gj:gj_voltage_solved_simple}
	V(\xi)
	&=
	 e^{-\xi D^T D^{T \dagger}} V(0)
	+
	\left(
		D^T D^{T \dagger}
	\right)^{-1}
	\left(
		I - e^{-\xi D^T D^{T \dagger}}
	\right)
	d.
\end{align}

Simplify the matrix $D^T D^{T \dagger}$ via its SVD:

\begin{align*}
	D^T
	&=
	V \begin{bmatrix} S \\ 0\end{bmatrix} \mathcal{U}^T,
	%
	\\
	\\
	%
	\implies
	D^{T \dagger}
	&= 
	\mathcal{U} \begin{bmatrix} S & 0\end{bmatrix} V^T
	%
	\\
	\\
	%
	\implies 
	D^T D^{T \dagger} 
	&=
	V \begin{bmatrix} S \\ 0\end{bmatrix} \mathcal{U}^T \mathcal{U} \begin{bmatrix} S & 0\end{bmatrix} V^T
	\\
	\\
	&=
	V \begin{bmatrix} S \\ 0\end{bmatrix} \begin{bmatrix} S & 0\end{bmatrix} V^T
	\\
	\\
	&= V \begin{bmatrix} I_{d} & 0 \\ 0 & 0 \end{bmatrix}  V^T
	\\
	\\
	&= \begin{bmatrix} I_{d} & 0 \\ 0 & 0 \end{bmatrix} \in \mathbf{R}^{N \text{ x } N},
\end{align*}
where $I_{d}$ denotes the d-dimensional identity matrix. Equation (\ref{eq:analysis:comparison_sc_vs_pcf_vs_gj:gj_voltage_solved_simple}) becomes
\\
$$
	V(\xi)
	=
	 e^{-\xi I_d } V(0)
	+
	\begin{bmatrix} I_{d} & 0 \\ 0 & 0 \end{bmatrix}^{-1}
	\left(
		I - e^{-\xi I_d}
	\right)
	d.
$$
The matrix $\begin{bmatrix} I_{d} & 0 \\ 0 & 0 \end{bmatrix}$ is not invertible. Consider instead only the first $d$ equations of the preceding system. 

$$
	V_j(\xi)
	=
	 e^{-\xi} V_j(0)
	+
	\left(
		1 - e^{-\xi}
	\right)
	d_j , \hspace{4mm} j = 1, \ldots, d.
$$
Note that $d_j = d^T c$, and consider neuron $j_{max} = j$ the first to reach the spike threshold. Recall its initial condition $v(0) = -\frac{||d||^2}{2}$ to arrive at
$$
	V(\xi)
	=
	 - e^{-\xi} \frac{||d||^2}{2}
	+
	\left(
		1 - e^{-\xi}
	\right)
	d^T c
$$
Compare with the corresponding PCF trajectory, equation (\ref{eq:analysis:comparison_sc_vs_pcf_vs_gj:const_stim:pcf_voltage_trajectory}). The preceding equation rearranges to
$$
	V(\xi) = 
	d^T c 
	- e^{-\xi}
	\left(
		d^T c  + \frac{||d||^2}{2}
	\right),
$$

which is identical to equation (\ref{eq:analysis:comparison_sc_vs_pcf_vs_gj:const_stim:pcf_voltage_trajectory}). Since only one neuron spikes, $r(\xi)$ and thus $\hat{x}(\xi)$ are identical for both PCF and gap-junction networks. 
The preceding, somewhat painful analysis shows that if the PCF and gap-junction models begin on their steady-state trajectories with the same initial conditions, their network estimates are identical in time. The statement is limited to the case of a constant driving stimulus. It does not, for example, show which network reaches the steady state trajectory first.




\end{enumerate}