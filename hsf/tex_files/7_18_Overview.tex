\section{The self-coupled SNN Model}

\textit{\textbf{Problem Statement}}
\\
\\
 Given:

\begin{itemize}
    \item A Linear Dynamical System  $\frac{dx}{dt} = A x(t) + B c(t)$,  $x \in \mathbf{R}^d$
    
    \item A Decoder Matrix $D \in \mathbf{R}^{d\hspace{1mm} \times \hspace{1mm}N}$ specifying the preferred directions of N neurons in d-dimensional space,
    
\end{itemize}
synthesize a spiking neural network that implements the linear dynamical system.   
\\
\\
\\
\textbf{\textit{Features}}
\\
\begin{enumerate}
    \item \textbf{\textit{Long-Term Network Accuracy}} The Deneve network assumes $\hat{x}=x$. We show this assumption produces estimation error between the network and its target system that increases with time. By avoiding this assumption, the self-coupled network remains accurate over time. 
    
    \item \textbf{\textit{Tuning Curve Rotation}} To most efficiently use neurons, we use orthogonal coding directions via SVD. The dynamics matrix $A$ is diagonalized by an orthonormal basis $\mathcal{U}$ in d-dimensional space, while the decoder matrix $D$ is chosen such that $\mathcal{U}$ gives its left singular vectors. This choice of coding directions eliminates connectivity between neurons with orthogonal encoding directions.
    
 	    
	At least two neurons per dimension (2d in total) are required since spikes, the encoding quanta, have positive unit-area.
    N-neuron ensembles can thus represent systems with $\frac{N}{2}$ dimensions or less. 
    
    \item \textbf{\textit{Post-synaptic Spike Dropping}} At each synapse, neurotransmitter release due to an action potential is probabilistic.  We incorporate probabilistic spike transmission by thinning at every synaptic connection. The pre-synaptic neuron's membrane potential is still deterministically reset by an action potential. 
    
    \item \textbf{\textit{Dimensionless Time}} We describe both the network and target system in dimensionless time. Time is normalized by the synapses' time constant, $\tau_s$. This dimensionless representation ensures consistent numerical simulation independent of simulation timestep. Furthermore, $\tau_s$ is implicitly specified as 1, reducing the model's parameters by one.     
\end{enumerate}

%\textbf{\textit{The Self-Coupled Equations}}
%\\
%Let the neurons have synaptic decay rate . Normalize the dynamical system to dimensionless time $\xi$:
%$$
%\xi \overset{\Delta}{=} 
%$$
%Assume the matrices $A$ and $D$ are full rank. Factor each via eigenvalue and singular-value decompositions (SVD) respectively. Rotate $D$ using its SVD so that $A$ and $D$ have the same left-eigenspace basis set, i.e:
%\begin{align*}
%A &= \mathcal{U} \Lambda \mathcal{U}^T\\
%\\
%D &= \mathcal{U} \begin{bmatrix}
%S & 0
%\end{bmatrix}
%V^T.
%\end{align*}
%The following quantities express network parameters in this rotated basis:
%\begin{enumerate}
%\item $\epsilon \in \mathbf{R}^{d}$ : The estimation error between the network estimate $\hat{y}$ and the true dynamical system state $y$
%$$
%\epsilon \overset{\Delta}{=} y - \hat{y}
%$$.
%\item $v \in \mathbf{R}^N$ : The membrane voltage of N neurons that compose the network.
%$$
%v \overset{\Delta}{=} S \epsilon
%$$
%\item $\tilde{o} \in \mathbf{R}^N$ :  The spike train of the neurons, represented as Dirac $\delta$-functions. 
%$$
%\tilde{o}_j \overset{\Delta}{=} \sum_{i=0}^{\infty} \delta(
%$$
%\end{enumerate}
%
%The self-coupled network implements the given dynamical system using a network of neurons whose membrane potential dynamics are given by 
%
%\begin{align}
%\label{overview:voltage_dynamics}
%\dot{v}
%&= 
%\begin{bmatrix}
%\Lambda & 0
%\\
%0 & 0
%\end{bmatrix}
%v +
%\begin{bmatrix}
%S \left(\Lambda + I_d \right) S & 0
%\\
%0 & 0
%\end{bmatrix}
%  \rho - 
% \begin{bmatrix}
%S^2 & 0
%\\
%0 & 0
%\end{bmatrix}
%    \tilde{o},
%\end{align}
%
%where 
%\begin{align*}
%\end{align*}