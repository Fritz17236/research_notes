\section{Derivation: The Predictive Coding Framework and Gap-Junction Network}


Here we derive the the predictive coding framework (PCF) as defined in Boerlin \& Deneve, 2013. We highlight an assumption in this model which we later show worsens the network estimate. The correction of this assumption produces a third model that directly coupled membrane voltages. We term this a model the \textit{gap-junction} (GJ) network.\\


\subsection{Identical Initial Derivation for All Three Models (PCF, GJ, SC):}
 The PCF and GJ derivations are initially identical to the SC model in section ({\ref{section:derivation:basic_model}). For all three models: 

\begin{enumerate}
    \item Let $\tau_s$ be the synaptic time constant of each synapse in the network. Define dimensionless time as:
    \begin{equation*}
        \xi \overset{\Delta}{=} \frac{t}{\tau_s}.
    \end{equation*}\\
    We now assume our Linear Dynamical System is expressed in dimensionless time, i.e
    
    \begin{equation}
        \label{eq:derivation:pcf_gj:lds_dimensionless}
        \frac{dx}{d\xi} = Ax(\xi) + B c(\xi).
    \end{equation}
    
    The neuron encoding directions are given by 
	$$
	D = \begin{bmatrix}
	d_1 & \hdots & d_N
	\end{bmatrix}.
	$$
	   
    To describe the neuron dynamics in dimensionless time, let $o(\xi) \in \mathbf{R}^{N}$ be the spike trains of N neurons composing the network with components
    \begin{equation*}
        o_j(\xi) = \sum_{k=1}^{\text{$n_j$ spikes}} \delta(\xi - \xi_{j}^{k}),
    \end{equation*}
    where $\xi_j^k$ is the time at which neuron $j$ makes its $k^{th}$ spike. 
    Define the network's estimate of the state variable as
    \begin{equation}
        \label{eq:derivation:pcf_gj:xhat}
        \hat{x}(\xi)
        \overset{\Delta}{=} D r(\xi),
    \end{equation}
    where $D \in \mathbf{R}^{d \times N}$ and 
    \begin{equation}
    \label{eq:derivation:pcf_gj:rdot}
        \frac{dr}{d \xi} = -r + o(\xi).
    \end{equation}\\

    The network estimation error is
    \begin{equation}
    \label{eq:derivation:pcf_gj:error_def}
        e(\xi) \overset{\Delta}{=} x(\xi) - \hat{x}(\xi).
    \end{equation}
    

    
	\item Each network greedily minimizes the objective 
	$$
	\mathcal{L}(\xi) = || x(\xi+d\xi) - \hat{x}(\xi + d\xi)||^2.
	$$
	
	When neuron $j$ spikes, the estimate becomes
	
	$$
	\mathcal{L}_{spike} = ||x - \hat{x} - d_j||^2.
	$$
	
	If $j$ does not spike the error is 
	$$	
   	\mathcal{L}_{ns} = ||x - \hat{x}||^2.
    $$
    
	Neuron $j$ spikes when it decreases the objective, i.e.
	
	$$
	\mathcal{L}_{sp} < \mathcal{L}_{ns},
	$$
	
	which gives spiking condition
	
	$$
	d_j^T e = \frac{||d_j||^2}{2}.
	$$
    
	This leads us to define voltage as 
	$$
		v\overset{\Delta}{=}  D^T e.	
	$$    
    
    \item From equations (\ref{eq:rdot}) and (\ref{eq:xhat}), we have
    
    \begin{align*}
        D \dot{r} + D r &= Do \\
        \\
        \implies \dot{\hat{x}} + \hat{x} &= Do,
    \end{align*}
    where the dot denotes derivative w.r.t dimensionless time $\xi$.


\end{enumerate}

\subsection{PCF Derivation:}
The voltage dynamics $v_{pcf}$ are given by 

\begin{align*}
\dot{v}_{pcf} &=
D^T \dot{e}
\\
\\
&= 
D^T \left(\dot{x} - \dot{\hat{x}}\right)
\\
\\
&=
D^T \left( A x + B c - Do - \hat{x}\right).
\end{align*}

The PCF argues that when the network functions, $x = \hat{x}$. We then have
\begin{align*}
\dot{v}_{pcf} &=
D^T \left( A \hat{x} + B c - Do - \hat{x}\right)
\\
\\
&= 
D^T \left( A - I\right)\hat{x} + D^T B c - D^T D o.
\end{align*}

With $\hat{x} = Dr$ PCF has voltage dynamics

\begin{align}
\label{eq:derivation:pcf_gj:voltage_dynamics_pcf}
\dot{v}_{pcf} &= D^T \left( A - I\right)D r + D^T B c - D^T D o.
\end{align}

Derive the spiking condition with an identical method to the SC network in section (\ref{section:derivation:basic_model}) to get PCF threshold voltages

$$
v_{th} = \frac{1}{2} \begin{bmatrix}
d_1^T d_1
\\
\vdots
\\
d_N^T d_N
\end{bmatrix}.
$$



\subsection{GJ Derivation:}

The voltage dynamics of the GJ model are 
\begin{align*}
\dot{v}_{GJ} &=
D^T \left( A x + B c - Do - \hat{x}\right).
\end{align*}

The GJ model does not assume $x = \hat{x}$. Rather, it uses equations (\ref{eq:derivation:pcf_gj:error_def}) and (\ref{eq:derivation:pcf_gj:voltage_def}):

\begin{align*}
v_{gj} &= D^T e
\\
\\
&= 
D^T \left(x - \hat{x}\right)
\\
\\
\implies 
x &= 
D^{T \dagger}v_{GJ}  + \hat{x}.
\end{align*}

Substitute this in the dynamics equation and simplify to get 
\begin{align}
\label{eq:derivation:pcf_gj:voltage_dynamics_gj}
\dot{v}_{GJ} &=
D^T A D^{T \dagger} v_{GJ}  + D^T\left(A-I\right)Dr  + D^T Bc - D^T D o.
\end{align}

The addition of the voltage coupling term $D^T A D^{T \dagger} v_{GJ}$ leads to the name Gap-Junction. 


\clearpage

