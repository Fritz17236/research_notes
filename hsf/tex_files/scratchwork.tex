
Given the network driving force $c(\xi) = \begin{bmatrix}
1 & 0
\end{bmatrix}$, 
we claim using an identical argument to the PCF case that only one neuron $j_{max}$ fires periodically in the gap-junction network. Namely, neuron $j_{max}$ receives a greater driving than all other neurons such that it reaches the threshold $v_{th} = \frac{1}{2}$ before any other neuron in the network. After the spike reset, it will reach the threshold first again so that the entire network reduces to periodic firing of neuron $j_{max}$. The dynamics

\begin{align*}
\dot{V}
&= 
D^T A
D^{T \dagger} V 
+
D^T
\left(
	A + I 
\right)
D r
+ 
D^T B c
- D^T D o,
\end{align*}
become the 1-d system

\begin{align*}
\dot{v}
&= 
\left(
D^T A
D^{T \dagger} V 
\right)
_{j_{max}}
+
D^T
\left(
	-I + I 
\right)
D r_{j_{max}}
+ 
D^T c_{j_{max}}
- \frac{1}{2} o_{j_{max}}
%
\\
\\
%
&= 
f(\xi) v
+
d^T c - \frac{1}{2} o,
\end{align*}

where $f(\xi)$ is a periodic function, $f(\xi) = D^T A D^{T \dagger} V(\xi)$.


\begin{align*}
\end{align*}
% simulate pcf network constant driving, compare to explicit equation
%To compute the per-spike RMSE of the PCF network,
%\begin{align*}
%	RMSE_{spike}^{pcf}(d) 
%	&=
%	\sqrt
%	{
%		\phi_{pcf}(d)	\int_{\tau = 0}^{\frac{1}{\phi_{pcf}(d)}}
%		e_{pcf}^T e_{pcf} (\tau) \, d \tau
%	},
%\end{align*}
%Assume the system is in steady state such that $x(\xi) = c(\xi)$. Then use $e_{pcf}(\tau) = x(\tau) - \hat{x}_{pcf}(\tau)$ to simplify the integral:
%\begin{align*}
%	e_{pcf}^T e_{pcf} &= 
%	x^T x 
%	-
%	2 x^T \hat{x}_{pcf}
%	+
%	\hat{x}_{pcf}^T \hat{x}_{pcf}
%	%
%	\\
%	\\
%	%
%	&= 
%	c^T c
%	- 2 c^T d 
%		\left(	
%			1 + 
%			\frac {1} {2 d^T c }
%		\right)
%		e^{-\tau}
%	+
%	||d||^2 
%	\left(
%		1 + \frac{1} {2 d^T c}
%	\right)^2
%	e^{-2\tau}
%	%
%	\\
%	\\
%	%
%	&=
%	1
%	- 
%	\left(
%		2 c^T d + 1
%	\right)e^{-\tau}
%	+
%	||d||^2
%	\left(
%		1 
%		+
%		\frac{1} {c^T d}
%		+ 
%		\frac{1} {4 (c^T d)^2}
%	\right)
%	e^{-2 \tau}.
%\end{align*}
%Integrate the preceding expression to get
%\begin{align*}
%	RMSE_{spike}^{pcf}(d) 
%	&=
%	\sqrt
%	{
%		\phi_{pcf}(d)	\int_{\tau = 0}^{\frac{1}{\phi_{pcf}(d)}}
%			1
%		- 
%		\left(
%			2 c^T d + 1
%		\right)e^{-\tau}
%		+
%		||d||^2
%		\left(
%			1 
%			+
%			\frac{1} {c^T d}
%			+ 
%			\frac{1} {4 (c^T d)^2}
%		\right)
%		e^{-2 \tau} \, d \tau
%	}
%	%
%	\\
%	\\
%	%
%	&=
%	\sqrt{
%		1
%		-
%		\phi_{pcf}
%		\left(
%			2 c^T d + 1
%		\right) 
%		\left(
%			1 - e^
%			{
%				-\frac
%				{
%					1
%				} 
%				{
%					\phi_{pcf}
%				}
%			}
%		\right)
%		+
%		\phi_{pcf} \frac{||d||^2}{2}
%		\left(
%			1 + \frac{1}{c^T d} + \frac{1}{4 (c^T d)^2}
%		\right)
%		\left(
%			1 - e^{-\frac{2}{\phi_{pcf}} }
%		\right)
%	}.
%\end{align*}
%We desire the RMSE as a function of spike rate. We assume that the tuning curve $d$ is sufficiently close to $c$ so that $c^T d = 
%\begin{bmatrix}
%1 & 0
%\end{bmatrix}
%\begin{bmatrix}
%d_1 \\ d_2
%\end{bmatrix}
%\simeq 
%d_1,
%$
%and that $||d||^2 \simeq d_1^2$. This lets us invert equation (\ref{eq:analysis:comparison_sc_vs_pcf_vs_gj:const_dynamics:pcf_spike_rate})  to obtain
%\begin{align*}
%-\frac{1} {\phi_{pcf}}
%&=
%ln
%\left(
%	d_1 - \frac{d_1^2}{2}
%\right)
%-
%ln 
%\left(
%	d_1 + \frac{d_1^2}{2}
%\right)
%%
%\\
%\\
%%
%\implies
%e^{-\frac{1}{\phi}}
%&= 
%\frac
%{
%	d_1 - \frac{d_1^2}{2}
%}
%{
%	d_1 + \frac{d_1^2}{2}
%}
%%
%\\
%\\
%%
%&=
%\frac
%{
%	1 - \frac{d_1}{2}
%}
%{
%	1 + \frac{d_1}{2}
%}
%%
%\\
%\\
%%
%\implies
%e^{-\frac{1}{\phi}}
% + 
%\frac{d_1 e^{-\frac{1}{\phi}}
%}{2}
%&= 
%1 - \frac{d_1}{2}
%%
%\\
%\\
%%
%\implies
%1 - e^{-\frac{1}{\phi}}
%&= 
%\frac{d_1}{2} 
%\left(
%	1 + e^{-\frac{1}{\phi}}
%\right)
%%
%\\
%\\
%%
%\implies
%d_1(\phi_{pcf})
%&= 
%2 \,
%\frac
%{
%	1 - e^{-\frac{1}{\phi}}
%}
%{
%	1 + e^{-\frac{1}{\phi}} 
%}.
%\end{align*}
%Use this to simplify our expression for $RMSE_{spike}^{pcf}$ above:
%
%\begin{align*}
%RMSE_{spike}^{pcf}(\phi_{pcf}) 
%	&=
%	\sqrt{
%		1
%		-
%		\phi_{pcf}
%		\left(
%			2 d_1 + 1
%		\right) 
%		\left(
%			1 - e^
%			{
%				-\frac
%				{
%					1
%				} 
%				{
%					\phi_{pcf}
%				}
%			}
%		\right)
%		+
%		\phi_{pcf} \frac{d_1^2}{2}
%		\left(
%			1 + \frac{1}{d_1} + \frac{1}{4 d_1^2}
%		\right)
%		\left(
%			1 - e^{-\frac{2}{\phi_{pcf}} }
%		\right)
%	}.
%\end{align*}
%
%Note that 
%
%\begin{align*}
%	e^{-\frac{1}{\phi_{pcf}}}
%	&=
%	\frac
%	{
%		d_1 - \frac{d_1^2}{2}
%	}
%	{
%		d_1 + \frac{d_1^2}{2}
%	}
%	%
%	\\
%	\\
%	%
%	&=
%	\frac
%	{
%		1 - \frac{d_1}{2}
%	}
%	{
%		1 + \frac{d_1}{2}
%	}
%	%
%	\\
%	\\
%	%
%	\implies
%	1 -e^{-\frac{1}{\phi_{pcf}}}
%	&= 
%	\frac
%	{
%		1 + \frac{d_1}{2} - 1 + \frac{d_1}{2}
%	}
%	{
%		1 + \frac{d_1}{2}
%	}
%	%
%	\\
%	\\
%	%
%	&= 
%	\frac{d_1}
%	{1 + \frac{d_1}{2}}
%	%
%	\\
%	\\
%	%
%	&=
%	\frac{1}
%	{
%		\frac{1}{2} + \frac{1}{d_1}
%	}.
%\end{align*}
%
%Furthermore, 
%
%\begin{align*}
%	e^
%	{
%		- 
%		\frac{2}{\phi_{pcf}}
%	}
%	&= 
%	\frac
%	{
%		\left(
%			d_1 - \frac{d_1^2}{2}
%		\right)^2
%	}
%	{
%		\left(
%			d_1 + \frac{d_1^2}{2}
%		\right)^2
%	}
%	%
%	\\
%	\\
%	%
%	&=
%	\frac
%	{
%		d_1^2 - d_1^3 + \frac{d_1^4}{4}
%	}
%	{
%		d_1^2 + d_1^3 + \frac{d_1^4}{4}
%	}
%	%
%	\\
%	\\
%	%
%	&=
%	\frac
%	{
%		1 - d_1 + \frac{d_1^2}{4}
%	}
%	{
%		1 + d_1 + \frac{d_1^2}{4}
%	}
%	%
%	\\
%	\\
%	%
%	\implies
%	1 - e^
%	{
%		- 
%		\frac{2}{\phi_{pcf}}
%	}
%	&=
%	\frac
%	{
%		1 + d_1 + \frac{d_1^2}{4}
%		-1 + d_1 - \frac{d_1^2}{4}
%	}
%	{
%		1 + d_1 + \frac{d_1^2}{4}
%	}
%	%
%	\\
%	\\
%	%
%	&=
%	\frac
%	{
%		2 d_1
%	}
%	{
%		1 + d_1 + \frac{d_1^2}{4}
%	}
%	%
%	\\
%	\\
%	%
%	&=
%	\frac
%	{
%		2
%	}
%	{
%	d_1^2
%	\left(
%		\frac{1}{ d_1^2}
%		+
%		\frac{1}{ d_1}
%		+ 
%		\frac{1}{4}
%	\right)
%	}
%	%
%	\\
%	\\
%	%
%	&=
%	\frac{2}
%	{
%		d_1^2 
%		\left(
%			\frac{1}{2} + \frac{1}{d_1}
%		\right)^2
%	}
%\end{align*}.
%
%Therfore the RMSE is 
%
%\begin{align*}
%RMSE_{spike}
%&=
%\sqrt{
%		1 - 
%		\phi_{pcf}
%\frac{2 d_1 + 1}
%	{
%		\frac{1}{2} + \frac{1}{d_1}
%}		
%		+
%		\phi_{pcf}
%\frac
%{		
%		\left(
%			1 + \frac{1}{d_1} + \frac{1}{4 d_1^2}
%		\right)
%}
%{
%		\left(
%			\frac{1}{2} + \frac{1}{d_1}
%		\right)^2
%}
%	},
%	%
%	\\
%	\\
%	%
%	&=
%	\sqrt{
%		1 - 
%		\phi_{pcf}
%\frac
%{
%d_1 + \frac{1}{2} + 2 + \frac{1}{d_1}	
%		+		
%			1 + \frac{1}{d_1} + \frac{1}{4 d_1^2}
%}
%{
%		\left(
%			\frac{1}{2} + \frac{1}{d_1}
%		\right)^2
%}
%	}
%	%
%	\\
%	\\
%	%
%	&=
%		\sqrt{
%		1 - 
%		\phi_{pcf}
%\frac
%{
%d_1 + 3.5 + \frac{2}{d_1}	
%	+ \frac{1}{4 d_1^2}
%}
%{
%		\left(
%			\frac{1}{2} + \frac{1}{d_1}
%		\right)^2
%}
%	}
%\end{align*}
