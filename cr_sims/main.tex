\documentclass{article}
\usepackage[utf8]{inputenc}
\usepackage{fullpage,enumitem,amsmath,amssymb,graphicx, caption, hyperref}
\usepackage{subcaption, multicol}
\newcommand\labelAndRemember[2]
  {\expandafter\gdef\csname labeled:#1\endcsname{#2}%
   \label{#1}#2}
\newcommand\recallLabel[1]
   {\csname labeled:#1\endcsname\tag{\ref{#1}}}

\hypersetup{
    colorlinks=true,
    linkcolor=blue,
    filecolor=magenta,   
    urlcolor=cyan,
}

\numberwithin{equation}{section}

\title{Stimulation of Plastic, Efficient Balanced Networks}
\author{Chris Fritz} 
\date{\today}

\begin{document}

\maketitle


\section{Purpose} Numerically Investigate Decoupling Stimulation Techniques on Efficient Balanced Networks with Plastic Weights. 

\section{Methods}


We use an Efficient Balanced Network with Synaptic Delays[Chalk, Deneve] to implement a linear dynamical system. We add Hebbian Spike-Timing-Dependent Plasticity [cite tass] into the synaptic connections of the resulting LIF network giving time-varying synaptic weights. We study the effect of coordinated-reset (CR) and random-reset (RR) stimulations on both the synchrony of the spiking activity measured by the Kuramoto order paramter [cite tass],  and the network's estimate of the given linear dynamical system. 



\begin{itemize}
\item \textit{\textbf{Efficient Balanced Networks:}} Using the Efficient Balanced Networks (EBN) Framework (cite deneve), we implement the linear dynamical system,
\begin{align}
\dot{x} &= 
\begin{bmatrix}  
0 & -1 \\
1 & 0
\end{bmatrix}
x, \notag
\\ 
\\ \notag
x(0) &= \begin{bmatrix}1 & 0 \end{bmatrix}.
\end{align}

The neuron's voltage leak and readout decay rates are  $\lambda_d = \lambda_V =  1$ respectively, and the neuron's encoding directions are equally distributed along the unit circle:
\begin{align}
D &= \begin{bmatrix} d_1 & \hdots & d_{64} \end{bmatrix}, \notag
\\
\\
d_j &=  \notag
\begin{bmatrix}
cos(\frac{j}{64} 2 \pi) &
sin(\frac{j}{64} 2 \pi) 
\end{bmatrix}^T.
\end{align}

The resulting LIF network with spike trains $o(t)$ has voltage dynamics

\begin{align}
\dot{v} &= -v + \Omega^{slow} \,  r - \Omega^{fast} \, o(t) + \eta(t) 
\\ \notag
\\ 
\dot{r} &= -r + o(t),
\\ \notag
\\\notag
\text{with }
\\\notag
\\
\Omega^{slow} &= D^T (A + I) D  \notag
\\\
\label{eq:kernels}
\\\notag
\Omega^{fast} &= D^T D,
\end{align} 
spiking thresholds 
\begin{equation}
v_{th} = \frac{||d_j||^2}{2} = 1,
\end{equation} and readout 
\begin{equation}
\hat{x} = D r. 
\end{equation}

\item \textbf{\textit{Synaptic Delays: }}


\item  \textbf{\textit{Spike-Timing Dependent Plasticity: }} We incorporate plasticity by dynamically altering the weights using Hebbian STDP rules.[cite cite] The network connectivity features two sets of synapses, $\Omega^{slow}$ and $\Omega^{fast}$ in equation (\ref{eq:kernels}). The  arrival of a spike from pre-synaptic neuron $i$ to post-synaptic neuron $j$ triggers an update to the synaptic weight between the pre and post-synaptic neurons $\Omega^{slow/fast}_{ij}$. The update to the weight is
$$
\Omega_{ij} \rightarrow \Omega_{ij} + W(t_j - (t_i + t_d))), 
$$

where

\begin{equation}
W(\Delta t) = \alpha  
\begin{cases}
e^{-\frac{|\Delta t|}{\tau_+}}, & \Delta t > 0
\\
0, & \Delta t = 0
\\
-\frac{\beta}{\tau_R} e^{-\frac{|\Delta t|}{\tau_-}},
\end{cases}
\end{equation}

models the time depedence of synaptic plasticity.

\item \textbf{\textit{Coordinated Reset Stimulation: }}

\end{itemize}





\begin{align}
1
\end{align}
where $S(t) \in \mathbf{R}^2$ is a stimulation signal. 
\\
\\
Model Synaptic Plasticity via STDP on synaptic connections between neurons. 
\\
Include Coordinated Reset and Random Reset Stimulation Techniques as Input Signal to EBN Networks
\\
\\
Study the resulting neuron voltage dynamics, the network's readout by: 
	
	 1) Measuring the Synchrony of the Neuron Voltages  
	 
	 2) Measuring the RMSE of the Network Readout vs The EBN 
	 
	 
	 
\clearpage

\end{document}

